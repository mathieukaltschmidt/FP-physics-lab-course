\documentclass[twocolumn,
			   showpacs,%
               nofootinbib,
               aps,%
               %eqsecnum,
               prd,
               notitlepage,
               showkeys,
               10pt]{revtex4-1}
               
%Loading my personal settings
\usepackage{todonotes}

%math and formulas
\usepackage{amssymb}
\usepackage{amsmath}
\usepackage{physics}

%language settings and microtype
\usepackage[english]{babel}
\usepackage{microtype}

%useful packages
\usepackage{graphicx}
\usepackage{siunitx}
\usepackage{xcolor}
\usepackage{float}
\usepackage{dcolumn}
\usepackage{blindtext}
\usepackage{xfrac}
\usepackage[labelfont=bf]{caption}
\usepackage{subcaption}


%feynman diagrams
\usepackage{tikz}
\usepackage{tikz-feynman}
\tikzfeynmanset{compat=1.0.0} 

%loading the `external` library
\usetikzlibrary{external}   
%Create directory to store the diagrams and all related files          
\immediate\write18{mkdir -p feynman-diagrams} 
% Activate externalization
\tikzexternalize[
  %Avoid cluttering the directory                     
  prefix=feynman-diagrams/, 
  %Calling lualatex externally
  system call={             
    lualatex \tikzexternalcheckshellescape -halt-on-error -interaction=batchmode -jobname="\image" "\texsource"  || rm "\image.pdf"
  },
]

%color settings
\definecolor{mydarkblue}{RGB}{1,1,141}

%correcting parindent
\setlength{\parindent}{0pt}


%always use hyperref at the end of the preamble!
\usepackage[colorlinks=True]{hyperref}
\hypersetup{allcolors=mydarkblue}

%main document
\begin{document}

%Personel data
\title{F91: Studying the $Z$ boson with the ATLAS Detector at the LHC }
\author{Mathieu Kaltschmidt}
\email{M.Kaltschmidt@stud.uni-heidelberg.de}
\affiliation{Heidelberg University,  D-69117 Heidelberg, Germany}
\author{Quirinus Schwarzenb\"ock}
\email{Schwarzenb\"ock@stud.uni-heidelberg.de}
\affiliation{Heidelberg University,  D-69117 Heidelberg, Germany}

\date[Carried out in the week of  ]{March 4$^{\text{th}}$, 2019}


\begin{abstract}
%This experiment has been performed as part of the advanced lab course for physics students (FP) at Heidelberg University.

\blindtext
\end{abstract}

\maketitle



\section{Introduction}
Just wanted to set up the \verb TikZ-feynman \normalfont \ package:

\begin{figure}[H]
\centering
	
\feynmandiagram [small] [layered layout, horizontal = a to b] {
a  -- [photon, edge label=\(Z^0\)] b ,
b -- [fermion] f1 [particle=\(l^{-}\)],
b -- [anti fermion] f2 [particle=\(l^{+}\)],
};
\caption{Leading order Feynman diagram for the $Z$ boson decay into lepton pairs, inspired by \cite{PPMasterclass}}
\end{figure}

Introducing Drell-Yan diagrams:
\begin{figure}[H]
\centering
	
\feynmandiagram [small] [horizontal' = a to b] {
i1 [particle=\(d\)] -- [fermion] a  -- [fermion] i2 [particle=\(\overline{d}\)],
a -- [photon, edge label=\(\gamma / Z^0\)] b ,
f1 [particle=\(\mu^{+}\)] -- [fermion] b -- [fermion] f2 [particle=\(\mu^{-}\)],
};
\caption{A simple Drell-Yan diagram, inspired by \cite{F91manual}}
\end{figure}

Already implementing the formulas mentioned in the theory part of \cite{F91manual} to speed things up a little.
\begin{align}
\mathcal{L} = \frac{N_1N_2f_{\text{rev}}n_b}{4\pi\sigma_x\sigma_y}
\end{align}

\blindtext

\begin{align}
\mathcal{L}_{\text{int}} = \int \mathcal{L} \ \dd t	
\end{align}

\blindtext

\begin{align}
N = \sigma_{pp\rightarrow X} \cdot \mathcal{L}_{\text{int}}
\end{align}


\section{Theory}

\blindtext

\subsection{This is a subsection}

\subsubsection{This is a subsubsection}

\blindtext





\section{Experiment}




\subsection{Experimental Setup}


\blindtext




\section{Results}
\blindtext



\begin{table}[!htbp]
\centering
\setlength{\tabcolsep}{2mm}
\renewcommand{\arraystretch}{1.5}
\begin{tabular}{|c||c|c|c|c|}
\hline
Test & $x_1$ [AU] & $x_2$ [AU] & $x_3$ [AU] & Quantity \\ \hline \hline
A & 2,2 & 3,8 & 6,2 & Q1 \\
\hline
B & 2,3 & 7,0 & - & Q2 \\
\hline
C & 2,1 & 11,6 & - & Q3 \\
\hline
D & 3,9 & 6,3 & - & Q4\\
\hline
E & 2,2 & 7,0 & - & Q5 \\
\hline
\end{tabular}
\caption{\label{tab:test}This is an example table.}
\end{table}




\section{Discussion}

\Blindtext


\begin{acknowledgments}

We would like to thank our supervisor Philipp Ott for his guidance throughout the operation of this experiment.

\end{acknowledgments}

\bibliographystyle{abbrv}
\bibliography{bibliography/literatur}
\nocite{*}

\end{document}
