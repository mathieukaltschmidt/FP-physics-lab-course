\chapter{Fundamental principles of astronomical measurements}
%TODO Detectors, observations, data reduction
\section{Detectors in Astronomy}

\section{Observations}

\section{Data Reduction}

\section{Basics of photometry}
What we are observing in our measurements is the \textbf{radiation flux $F$} of the stars which is given by
\begin{align}
	F = \frac{L}{4\pi d^2}
\end{align}
where $d$ is the distance between the observer and the star and $L$ is his luminosity. \\
To normalize our measured values we are using the sun as reference for the units we choose. For example the luminosity of the sun is 
	$1 \ L_{\odot} = 3.846 \cdot 10^{26} \ \si{\watt}$. \\	
The \textbf{Stefan-Boltzmann law} explains the connection between the surface temperature of a star and his flux. It is given by
\begin{align}
	F = \sigma \ T_{\text{eff}}^4\label{eq1.2}
\end{align}
with the Stefan-Boltzmann constant $\sigma = 5.67 \cdot 10^{-8} \ \si{\watt\meter^{-2}\kelvin^{-4}}$. To understand the meaning of the effective temperature $T_{\text{eff}}$ we need the know what we understand as a black body. A black body is, in theory, an object that absorbs every radiation independent on the corresponding wavelength and does'nt reflect any of it. The effective temperature in (\ref{eq1.2}) is the temperature a black body with the same surface as the star would need to emit the same radiation power. \\
We already introduced a new quantity, the luminosity of a star which is defined as the surface area $A$ times the flux. From this we can derive a relation between luminosity and temperature using (\ref{eq1.2}):
\begin{align}
	L = 4\pi R^2\sigma T_{\text{eff}}^4
\end{align}
with the radius $R$ of the star.

\section{Magnitudes}
We need to introduce different magnitude scales due to the fact that every instrumental setup is different but still we need to compare results taken from various observations.
\subsection{Instrumental Magnitude}

\subsection{Apparant Magnitude}

\subsection{Absolute Magnitude}