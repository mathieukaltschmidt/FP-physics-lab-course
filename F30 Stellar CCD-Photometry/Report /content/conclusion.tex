\chapter{Discussion of our results}
In this chapter we want to conclude our results and discuss possible reasons for deviations from what we would have expected. \\
\begin{description}
	\item[Characteristics of the CCD:] We determined the band gap of silicon by verifying the temperature dependency of the detected counts we expected from equation (\ref{eqn: dark current}). Our result $E_g = (1.259 \pm 0.005)$ eV is slightly higher the reference value for silicon ($1.15$ eV). This can have different reasons. \\ One possible explanation for the deviation is, that the  result could be mainly affected by the  measurements in the beginning  of the cooling process, where we still had a lot of noise because of thermal activation. On top, the value for the band gap is not unique because it is still dependent on the temperature range we are observing. We worked on a large range of about 100 degrees. This means we have to take into account that the value for the band gap is not perfectly constant during all our measurements. \\ Another possible reason is that the silicon  used in our measurement is not pure silicon or  already shows some wear. \\ We repeated the temperature measurement every 30 seconds for the whole time of the cooling process. This should be enough to obtain sufficient precision and statistics. \\
		We determined and subtracted an individual bias for every measurement to avoid an underestimation of the errors in the first part of the measurement and to present our results as general as possible. \\
		 We could have done more measurements in some specific temperature ranges to find out how much the band gap varies to confirm the problem we mentioned above. 
	\item[Flat field corrections:] We performed flat field corrections based on the determination of a normalized master flat field for every of the filter constellations. \\ As already discussed in the evaluation, it is quite useful to work in relative units when we want to have a good comparability of the results.\\ The impact of this correction method has impressively be shown by comparing the same exposure before and after dividing it by the master flat. The quality of the image improved a lot as one can see in figure (\ref{fig:ffc}). \\
	The three histograms presented in figure (\ref{fig:histograms}) showed the expected distribution around the value 1 because this value represents the mean intensity which we normalized the images to. The effect of the dead pixels which do not contribute to the measurement is also clearly represented by the large peaks at zero. 
	\item[Linearity and Dynamical range:] In this part of the experiment we analyzed the linear domain of the measure electronics. Unfortunately we did not take enough measurements at higher integration times to find the expected plateau at the saturation level. \\
	Nevertheless we were still able to see the almost perfectly linear relationship for all our measured domain. This conclusion is supported by the calculation of the corresponding $R^2$ values as explained before. \\
	If we could repeat the measurement we would definitely take more values at high integration times to be really able to find the limitations of the electronics. At least we can conclude that our measurements shouldn't be affected too much by unwanted saturation. 
	\item[Sensitivity and noise properties:] In this part we learned how much our results are affected by the noise properties and the sensitivity of the experimental setup. \\
	We found out that the PRNU noise dominates the other noises presented in this section. This is due the fact that we are working with high intensities.\\
	The electron gain $\kappa$ has been determined to: $\kappa = (2.297 \pm 0.025)$ from the fit. This value was not directly comparable with the value we got from calculating $\kappa$ by hand, using equation (\ref{eqn: kappa}). \\
	The read out noise $\sigma_R = (9.2 \pm 5.1)$ is small compared to the other error sources. This leads us to the conclusion that the precision of the electronics is sufficient for our case.
	\item[Globular Cluster BS90$^{\mathbf{14}}$:] In the final part of the experiment we worked on data concerning the globular cluster BS90$^{\mathbf{14}}$. \\
	First we needed to find the zero point used for calibration of the magnitudes. Our determination was based on ten values for reference stars from the \texttt{SIMBAD} catalogue. To achieve a higher accuracy we could have taken more references into account. Our estimated errors are mainly based on statistical properties. Especially for this case one should take care of a large amount of reference values to find a useful estimate. \\
	The same holds for the next step, where we chose around 25 stars to be able to perform the PSF fitting. We repeated this selection three times to find the best possible result for \texttt{STARFINDER}s fitting routine. \\
	After matching the stars found in both exposures we finally obtained the Color-Magnitude diagram which gave us the possibility to access the information about age and metallicity of the cluster. \\
	The results presented in table (\ref{tab: CMD}) vary a lot compared to the reference value which is represented by the green isochrone. Especially the value for the metallicity is significantly different. It was nearly impossible to try out every possible combination of the parameters. That's the reason why we chose three different curves that matched quite good to be able to show how sensible these fitting routines are and how hard and ambiguous it is to find the \textit{correct} values.\\ 

\end{description}
In general we can conclude that we learned a lot about astronomical research, especially about data analysis and image processing. 