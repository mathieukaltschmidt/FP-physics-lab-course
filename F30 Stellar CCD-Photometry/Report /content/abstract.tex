\makeatletter

\begin{centering}
\textbf{\Large\@title} \\
\vspace{.1cm}
\small\@author \\
\end{centering}

\makeatother

\vfill
%Abstract in english language
 
{\let\raggedsection\centering
\begin{centering}
\section*{Abstract }
\end{centering}
This experiment has been performed as part of the advanced lab course for physics students (FP) at Heidelberg University. \\
The theoretical and experimental basic knowledge which is  needed for the understanding of the conducted measurements is introduced and important concepts of modern astronomical research are presented and discussed.\\
The characteristics of a CCD-camera are analyzed using data we recorded with the KING telescope at MPIA. \\
Furthermore important astronomical data analysis tools used in current research are introduced and applied to our measurements. \\
Lastly we use an image of the globular cluster BS90$^{14}$ taken by the Hubble Space Telescope (HST) to generate a Color-Magnitude Diagram in order to determine age and metallicity of the cluster.


\vfill
%Abstract in german language
\begin{german}
\begin{centering}
\section*{Zusammenfassung}
\end{centering}
Dieses Experiment wurde im Rahmen des Fortgeschrittenen-Praktikums für Studierende der Physik (FP) an der Universität Heidelberg durchgeführt. \\
Es werden die theoretischen und experimentellen Grundlagen zum Verständnis der durchgeführten Messungen vorgestellt und grundlegende Konzepte moderner astronomischer Forschung erläutert und diskutiert. \\
Die charakteristischen Eigenschaften einer CCD-Kamera werden am Beispiel unserer Messungen mit dem KING-Teleskop am MPIA überprüft und wichtige Elemente der astronomischen Datenanalyse eingeführt. \\
Am Beispiel einer Aufnahme des Kugelsternhaufens BS90$^{14}$, welche vom Hubble Space Telescope (HST) aufgezeichnet wurde wird ein Farben-Helligkeits-Diagramm erstellt und damit das Alter und die Metallizität des Sternhaufens ermittelt. 

\end{german}}
\vfill
\blankpage